\documentclass[12pt,a4paper]{article}
\usepackage[utf8x]{inputenc}
\usepackage{ucs}
\usepackage{amsmath}
\usepackage{amsfonts}
\usepackage{amssymb}
\usepackage{graphicx}
\usepackage{hyperref}
\usepackage{float}
\usepackage[brazil]{babel} % Comentário, Português do Brasil
\author{Danilo Barbosa Oliveira - 9266671}
\title{Anotação de Algebra Linear}

\begin{document}
\maketitle
\newpage

\tableofcontents
\newpage

\section{Aula 14 de Agosto de 2018} 


\subsection{Sistemas Lineares}

Um sistema linear com n-equações e n-variáveis (digamos $x_1$, $x_2$,$x_n$) com coef. em F, é um conjunto de equações do tipo:

\begin{align*}
a_{11}x_1 + a_{12}x_2
\end{align*}

Resolver um sistema linear S é encontrar n-elementos de $F_1$, digamos $\alpha_1$, $\alpha_2$ ... $\alpha_n$ que satizfaz cada equação de S, ou seja:



Um sistema linear que admite solução é chamado de \textbf{compatível} (ou constante). Caso contrário, é chamado \textbf{incompatível}.

\begin{align*}
a_{11}\alpha_1 + a_{12}\alpha_2 + a_{mn}\alpha_n &= b_1\\
a_{21}\alpha_1 + a_{22}\alpha_2 + a_{mn}\alpha_n &= b_s\\
a_{m1}\alpha_1 + a_{m2}\alpha_2 + a_{mn}\alpha_n &= b_m
\end{align*}

Quando m=n e A 

\begin{align*}
AX &= B\\
A^{-1}(AX) &= A^{-1}B\\
(A^{-1}A)X &= A^{-1}B\\
(Id_n)X &= A^{-1}B\\
X &= A^{-1}B
\end{align*}

\paragraph{Regra de Cramer}

Considere o sistema S com $m=n  AX=B$ ; 

\begin{equation}
b_i = \sum_{k=1}^{n} a_{ik} x_k, 1 <= i <= n
\end{equation}

\paragraph{Matriz Amplificada (ou Ampliada) do Sistema Linear}

Dado o sistema linear:

\begin{align*}
2x + 3y = \sqrt{2}\\
5x - \sqrt{3}y = 1
\end{align*}

Na forma de matrix amplificada, temos:

\[
\begin{pmatrix}
2 & 3 & | & \sqrt{2}\\
5 & -\sqrt{3} & | & 1
\end{pmatrix}
\]


\paragraph{Método Básico}

Vamos efetuar 3 tipos de operações básicas (chamam operações elementares sobre as linhas da matriz ampliada):

\begin{enumerate}
	\item Trocar linhas da matriz amplificada;
	\item Multiplicar uma linha inteira por uma constante não nula;
	\item Somar uma linha com um múltiplo de uma linha;
\end{enumerate}

Exemplo:

\begin{align*}
3x + 2y - z &= -5\\
5x + 3y + 2z &= 1\\
3x + y + 2z &= 0\\
-x -y + 4z &= 11
\end{align*}

A Matriz ampliada fica:

\[
\begin{pmatrix}
3 & 2 & -1 & | & -5\\
5 & 3 & 2 & | & 1\\
3 & 1 & 2 & | & 0\\
-1 & -1 & 4 & | & 11
\end{pmatrix}
\]

Trocando a linha 1 ($L_1$) pela linha 2 ($L_2$), temos:

\[
\begin{pmatrix}
5 & 3 & 2 & | & 1\\
3 & 2 & -1 & | & -5\\
3 & 1 & 2 & | & 0\\
-1 & -1 & 4 & | & 11
\end{pmatrix}
\]

Fazendo a mesma coisa com a matriz identidade:
\begin{align*}
\begin{pmatrix}
0 & 1 & 0 & 0\\
1 & 0 & 0 & 0\\
0 & 0 & 1 & 0\\
0 & 0 & 0 & 1\\
\end{pmatrix}\times
\begin{pmatrix}
3 & 2 & -1 & | & -5\\
5 & 3 & 2 & | & 1\\
3 & 1 & 2 & | & 0\\
-1 & -1 & 4 & | & 11
\end{pmatrix}=
\begin{pmatrix}
5 & 3 & 2 & | & 1\\
3 & 2 & -1 & | & -5\\
3 & 1 & 2 & | & 0\\
-1 & -1 & 4 & | & 11
\end{pmatrix}
\end{align*}

\section{Aula 24 de Agosto de 2018}
\subsection{Combinação Linear}




\end{document}






